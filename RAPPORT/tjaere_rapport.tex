% MED UDGANGSPUNKT I ABSALON-FILEN:
\documentclass[a4paper,%                                      A4 papir st�rrelse
               reprint,%                                      preprint er mere overskueligt mens man skriver, reprint er det endelige format
               aps,%                                          dokument f�r layout som givet af APS
               prl,%                                          mere layout
               amsfonts,%                                     load AMS fonts
               amssymb,%                                      load flere AMS symboler
               amsmath,%                                      load AMS matematik (equation environment osv.)
              % nobibnotes,%                                   hvis man ikke bruger BibTeX b�r REVTeX f� at vide at den skal bruge normale fodnoter
               twoside,%                                      to-sidet print
               balancelastpage,%                              balancering af sidste side n�r der er to kolonner
               eqsecnum]%                                     nummerering af ligninger med section
               {revtex4-1}
\def\andname{og}                                              % Omd�ber "and" fra REVTeX style filen til dansk "og" (fjern linie hvis I skriver p� engelsk)
                    
\usepackage[utf8x]{inputenc}
\usepackage[english]{babel}   
                                 % Kapitler mv. f�r rigtige navne til sproget (ret til english hvis I skriver p� engelsk)
\usepackage[T1]{fontenc}                                      % Dansk tastatur

\usepackage{graphicx}
\setcounter{secnumdepth}{1}
\usepackage{array}
\usepackage{multirow}
\usepackage{longtable}
\usepackage{gensymb}
\usepackage{url}
\setlength{\parindent}{0pt}
 % \usepackage[
 % top    = 2.50cm,
 % bottom = 2.50cm,
 % left   = 1.5 cm,
 % right  = 1.5 cm]{geometry}
% Her s�tter I alle andre pakker I har brug for, f.eks. graphicx, xcolor, siunitx osv.

\begin{document}                                         % Starter det egentlige dokument

\title{Working title Tjæreborg}
\date{\today}                                                 % Dato p� dokument, brug evt. bare \today
\author{Johannes de Fine Licht}                                    % Forfatter p� artiklen (brug en ny \author{} kommando for hver forfatter)
\author{Mads Holst Aagaard Madsen}  
\author{Malte Lyndby Schmidt}       
\affiliation{ \bf{ Applied statistics; Project 2}}                        % Forfatternes tilh�rsforhold

\begin{abstract}                                              
\input{abstract.tex}
\end{abstract}

\maketitle                                                    % Laver overskrift mm. for artiklen

\section{Introduction}
\label{sec:introduction}
We have aquired a set of wind speed measurements from different heights. 

A number of different models exist to describe the average wind speed $U$ as a
function of the height $z$

$$
U(z) = \left( \frac{U_0^*}{\kappa} \right) \ln \left( \frac{z}{z_0} \right)
$$

We wish to determine which of these models gives the best fit for our data. 


% vi har data, præsenter
% hvad ved andre om data?

% hastighed som fkt. af højde

% der skal være noget kolmogorov
% spectral density

% coherens plots



% nyquist sampling skal være i diskussion


\section{Method}
\label{sec:method}
\input{method.tex}

\section{Results}
\label{sec:results}
The results are divided in two subsections just as the data: the \textbf{Anemometer} and the \textbf{Sonic Anemometer}.
All plots and results will be discussed in the \textbf{Discussion}. The data analysis has been carried out twice: 
once in Matlab \cite{matlab} (its fast) and once in Python \cite{python}/ ROOT \cite{root}. 
The plots are from the latter except the Coherence plot, figure (\ref{fig:Coherence_ANE}).

% RAW-data ---------------------------------------------------------------------------------------------------------------

\subsection{Anemometer}
In figure (\ref{fig:raw_ANE}) the raw data from the Anemometer along with the mean is plotted from six different heights. 

\begin{figure}
\centering
\includegraphics[width=0.5\textwidth]{Figurer/rawANE.pdf}
\caption{Raw signal data from all 6 heights measured with the Anemometer.}
\label{fig:raw_ANE}
\end{figure}

% Coherence ----------------------------------------------------------------------------------------------------------

To analyse the relation between Anemometers at different heights, we calculate the coherence between to Anemometers. An example is given in figure (\ref{fig:coherence_ANE}).

\begin{figure}
\centering
\includegraphics[width=0.5\textwidth]{Figurer/coherenceANE.pdf}
\caption{Coherence between 90 m and 57 m. The plot is normalised, which means the x-axis range is $[0 ; 1]$ instead of real frequencies in range $[0 \textrm{Hz} ; 17.5 \textrm{Hz}]$. In the figure only the lowest $1.5\%$ is plotted.}
\label{fig:coherence_ANE}
\end{figure}
 



% Fourier ----------------------------------------------------------------------------------------------------------

The data is then transformed, using a Fast Fourier Transformation, FFT, as shown in figure (\ref{fig:Fourier_ANE}).

\begin{figure}
\centering
\includegraphics[width=0.5\textwidth]{Figurer/FourierANE.pdf}
\caption{Fast Fourier Transformation of the rawdata from 6 heights.}
\label{fig:Fourier_ANE}
\end{figure}


% Fourier FIT - figure and table -------------------------------------------------------------------------------------- 

The data from figure (\ref{fig:Fourier_ANE}) is then plotted in a power spectrum with logarithmic axis  as in figure (\ref{fig:FourierFIT_ANE}), and fitted to a power function of the type; $f(x) = a \cdot x^{b}$. The resulting slope parameters, $b$, for all six heights are listed in table (\ref{tab:RES_ANE}).


\begin{table}[htbp!]
  \centering
  \begin{tabular}{lrr}
    \hline
    \hline
    Height           &$b$ value     &Uncertainty      \\
    \hline
    $90 $ m         &$1$ m      &$0.1\cdot 10^{-2}$ m       \\
    $77 $ m          &$1$ m     &$0.1\cdot 10^{-2}$ m         \\
    $57 $ m          &$1$ m     &$0.1\cdot 10^{-2}$ m         \\
    $42 $ m          &$1$ m    &$0.1\cdot 10^{-2}$ m         \\
    $28.5 $ m          &$1$ m      &$0.2\cdot 10^{-2}$ m         \\
    $17 $ m          &$1$ m      &$0.2\cdot 10^{-2}$ m         \\
    \hline
    \hline
  \end{tabular}
  \caption{The fitted values for the slope parameter in the power spectrum along with the uncertainty.}
  \label{tab:RES_ANE}
\end{table}

\begin{figure}
\centering
\includegraphics[width=0.5\textwidth]{Figurer/FourierFIT_ANE.pdf}
\caption{Fit in a power spectrum from 1 height; 90 m.}
\label{fig:FourierFIT_ANE}
\end{figure}

\section{Discussion}
\label{sec:discussion}
\input{discussion.tex}

\section{conclusion}
\label{sec:conclusion}
\input{conclusion.tex}


\begin{thebibliography}{99}                                   

% Litteratur der kan henvises til - for at opdatere listen nedenfor skal bbl-filen slettes inden der compiles!
  \bibitem{barlow} Roger J. Barlow; \emph{Statistics: a guide to the use of
     statistical methods in the physical sciences}, (Wiley, 1989) 


\end{thebibliography}


\end{document}                                            
